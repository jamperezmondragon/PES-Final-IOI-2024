\documentclass[12pt]{scrartcl}
\usepackage{config}
\usepackage{minted}

%\newcommand\mrh{\color{white}\bfseries}
\newcommand\mrc[1]{\begin{tabular}{@{}l@{}} #1 \end{tabular}}
\setlength\arrayrulewidth{0.8pt}

\usemintedstyle{pastie}

\begin{document}
    \hh{Cynthia Game}
    
    {\itshape This is a problem}
    
    \vspace{10pt}

    \hh{Problem}
    
    There is a round table with $1 \le N \le 16$ barrels, equidistant around the edge of the table. Each barrel has a coin inside, which can be heads or tails. The initial position of the coins is chosen by Cynthia. Cynthia challenges Juan to a game. In one move, Juan chooses the positions of some of the barrels and tells Cynthia. Then, Juan closes his eyes, and Cynthia rotates the table so that the positions are rotated as much as she wants. Afterward, Cynthia flips the coins inside the barrels in the positions chosen by Juan. Juan never knows the state of the coins, nor how much Cynthia rotates the table in her moves. Juan wins if at any moment all the coins are tails up. Cynthia wins if Juan does not win after $2^N$ moves. You must choose, for different values of $N$, which player to play with, and devise a winning strategy.

    \hh{Implementation Details}

    You must implement the function \textit{Decision()}, which receives an integer $N$, and returns an integer $K$. $K$ must be equal to 1 if you decide to play with Juan, and 0 if you decide to play with Cynthia. Any other value of $K$ will result in a wrong answer verdict. You must implement two functions, each with the strategy of one of the players.
    
    The function \textit{Cynthia()} receives an integer $N$, and can call the function \textit{Inicializa()}, which receives a vector of $N$ integers (must be 0 or 1), the initial states of the coins (a 1 represents a heads up coin, and a 0 tails up). And returns a vector of $N$ integers (all 0 or 1), the first move of Juan (a 1 represents that the coin in that position changes state). For the next moves, it can call the function \textit{Turn()}, this function receives an integer $0 \le x < N$, and rotates the table to the right by the factor $x$. Returns a vector with $N$ integers, describing Juan's next move. This vector can consist of values equal to 1 or 0, or consist only of $N$ values equal to -1. This vector represents that Juan has given up. To carry out the interaction, you must include in your program the library \textit{``Juego.h"}, with the command \textit{\#include ``Juego.h"}.

    The function \textit{Juan()} implements Juan's strategy. It receives an integer $N$, and must return a vector of vectors of ints, each vector of ints must consist of $N$ elements equal to $1$ or $0$, and represent the positions that change state, and those that do not, in each of his moves.

    An example of what the program would look like is the following:

    \begin{minted}{c++}
        #include "Juego.h"
        #include <bits/stdc++.h>
        using namespace std;

        int Decision(int N) {
            // Implement this function.
        }

        void Cynthia(int N) {
            // Implement this function.
        }

        vector<vector<int>> Juan(int N) {
            // Implement this function.
        }
    \end{minted}

    Consider that the grader will call the function \textit{Decision()} \textbf{multiple} times for each test case.

    \hh{Example}

    {\itshape Example 1:}

        The grader calls the function \textit{Decision(2)}. Let's imagine that your program returns:

        \begin{itemize}
            \item Let's say your program returns 0. Then, the grader calls the function \textit{Cynthia(2)}. Then, your code could call the function \textit{Inicializa(\{1, 0\})}. In this case, there would be one heads up coin, and another tails up. 
            Then, your code calls the function \textit{Turn(1)} or \textit{Turn(0)}, receiving the vectors that represent Juan's moves, until one of the two function calls responds with the vector \textit{\{-1, -1\}}. Then the game will end, and you must terminate the program.
            \item On the other hand, let's say your program returns 1. That is, it chose to play with Juan. Then the grader would call the function \textit{Juan(2)}. An example of a vector of vectors that it could return is the following:

            \begin{center}
                {\itshape \{\{1, 0\}, \{1, 1\}, \{0, 1\}, \{1, 0\}\} }
            \end{center}

            Depending on whether this array allows Juan to win with Cynthia playing optimally or not, the jury will award you the corresponding points.
            
        \end{itemize}

    \hh{Considerations}

        \begin{itemize}
            \item $1 \le N \le 16$.
            \item The maximum number of moves in both cases is $2^N$.
            \item Not following the implementation guidelines will be awarded 0 points.
            \item It is guaranteed that the jury always executes a winning strategy.
        \end{itemize}

    \hh{Subtasks}
    
        \begin{itemize}
            \item (10 points) $N \le 3$.
            \item (20 points) $N \le 6$.
            \item (30 points) $N \le 10$.
            \item (40 points) $N \le 16$.
        \end{itemize}

        Additionally, for each subtask, if you correctly solve all cases where Juan has a winning strategy, you will receive 70\% of the points it represents.
\end{document}
