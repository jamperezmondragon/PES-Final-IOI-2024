\documentclass[12pt]{scrartcl}
\usepackage{config}
\usepackage{minted}

%\newcommand\mrh{\color{white}\bfseries}
\newcommand\mrc[1]{\begin{tabular}{@{}l@{}} #1 \end{tabular}}
\setlength\arrayrulewidth{0.8pt}

\usemintedstyle{pastie}

\begin{document}
    \hh{Cynthia Juego}
    
    {\itshape Este es un problema de Solo Salida}
    
    \vspace{10pt}

    Hay una mesa redonda con $1 \le N \le 16$ barriles, equidistantes al rededor del borde de la mesa. Cada barril tiene una moneda dentro, que puede estar boca arriba o boca abajo. La posición original de las monedas las elige Cynthia. Cynthia reta a Juan a un juego. En un movimiento, Juan elige las posiciones de algunos de los barriles y se los dice a Cynthia. Luego, Juan cierra los ojos, y Cynthia gira la mesa de tal forma que las posiciones se rotan tanto como ella quiera. Después, Cynthia voltea las monedas dentro de los barriles en las posiciones que eligió Juan. Juan nunca sabe el estado de las monedas, ni cuanto rota la mesa Cynthia en sus movimientos. Juan Gana si en algún momento, todas las monedas están boca abajo. Cynthia gana, si Juan no gana después de $2^N$ movimientos. 
    Debes elegir, para distintos valores de $N$, con qué jugador jugar, e idear una estrategia ganadora.
    
    \hh{Problema}

    Para cada subtarea del problema, hay dos casos. El primer caso, tiene descrita la estrategia de Juan. 

    La primera línea contiene el valor de $N$, la segunda línea contiene la cantidad de movimientos en su estrategia $S$, y las siguientes $S$ líneas contienen una secuencia de $1 \le K \le N$ enteros $0 \le x_i < N$ separados por espacios: las posiciones de los barriles que voltea en ese movimiento.

    Para este tipo de casos, el archivo de salida debe contener una estrategia con la que Cynthia puede asegurar ganar ante la estrategia de Juan.

    Se garantiza que esto siempre es posible.

    Los otros casos, son archivos con un solo número, el valor de $N$. En estos casos, el archivo de salida debe contener una estrategia para Juan, con la que pueda asegurar ganar, independientemente de cómo juegue Cynthia.
    
    Se garantiza que esto no siempre será posible. 

    
    \hh{Detalles de salida} 

        En los casos del primer tipo, la primera línea de tus archivos debe contener $1 \le K \le N$ enteros positivos separados por espacios: las posiciones de las monedas que inicialmente están boca arriba. La siguiente línea, debe contener un entero $M$, la cantidad de movimientos. Luego, las siguientes $M$ líneas deben contener un entero positivo $1 \le x \le N$, el factor de rotación en dicho movimiento. 

        En los casos del segundo tipo, tus archivos deben contener un entero $M$, la cantidad de movimientos. Luego, las siguientes $M$ líneas deben contener $1 \le K \le N$ enteros $0\le x_i < N$ separados por espacios: las posiciones de los barriles que volteas en ese movimiento.

    \eject
    
    \hh{Evaluación}

    Para cada caso, si tu archivo de salida no resuelve el problema de dicho caso, recibirás 0 puntos. 
    
    La cantidad de puntos que da el caso del primer tipo de la subtarea $i$ es $a_i$, y la cantidad de puntos que da el caso del segundo tipo es $b_i$. Estos valores están escondidos, y algunos podrían ser 0. En la siguiente tabla, damos los valores de $N$ y $a_i + b_i$ para cada subtarea.
    
    

    \begin{center}
        \begin{tabular}{|c|c|c|}
            \hline
            Subtarea & Valor de $N$ & Puntaje total de la subtarea ($a_i + b_i$) \\
            \hline
            1 & $N = 2$ & 5 puntos.  \\
            \hline
            2 & $N = 3$ & 2 puntos. \\
            \hline
            3 & $N = 4$ & 10 puntos. \\
            \hline
            4 & $N = 5$ & 4 puntos. \\
            \hline
            5 & $N = 6$ & 5 puntos. \\
            \hline
            6 & $N = 7$ & 4 puntos. \\
            \hline
            7 & $N = 8$ & 20 puntos. \\
            \hline
            8 & $N = 10$ & 5 puntos. \\
            \hline
            9 & $N = 12$ & 5 puntos. \\
            \hline
            10 & $N = 14$ & 5 puntos. \\
            \hline
            11 & $N = 15$ & 5 puntos. \\
            \hline
            12 & $N = 16$ & 30 puntos. \\
            \hline
        \end{tabular}
    \end{center}
    
        

    
    
\end{document}
