\documentclass[12pt]{scrartcl}
\usepackage{config}
\usepackage{minted}

%\newcommand\mrh{\color{white}\bfseries}
\newcommand\mrc[1]{\begin{tabular}{@{}l@{}} #1 \end{tabular}}
\setlength\arrayrulewidth{0.8pt}

\usemintedstyle{pastie}

\begin{document}
    \hh{Doblemente Aficionados}
    
    
    \vspace{10pt}

    
    \hh{Problema}
    
        Dado un entero positivo $N$, considera un arreglo $a$ de tamaño $N$. Determina el tamaño del subarreglo \footnote{Un subarreglo es un arreglo que resulta de eliminar algún prefijo o sufijo del arreglo original.} de $a$ más grande, que tiene al menos dos valores con frecuencia máxima. 
    
    \hh{Detalles de Implementación}

       Debes implementar la función $Doblemente\_Aficionados()$. Esta función recibe un entero $N$ y un vector $a$, con $N$ elementos. Esta función debe regresar un entero, el tamaño máximo de un subarreglo de $a$ que tenga al menos dos valores con frecuencia máxima.
        La función se vería así:

\begin{minted}{c++}
#include <bits/stdc++.h>
using namespace std;

int Doblemente_Aficionados(int N, vector<int> a) {
    // Implementa esta función.
}
\end{minted}

    El evaluador correrá la función \textbf{múltiples} veces en cada caso de prueba.

    \hh{Ejemplos}
    
        {\itshape Ejemplo 1:}
        \begin{itemize}
            \item El evaluador llama la función 

            \begin{center}
                {\itshape Doblemente\_Aficionados(6, \{1, 1, 2, 2, 1, 5\})}
            \end{center}
            
            \item en este caso, regresar $5$, daría un veredicto de aceptado, pues el subarreglo $a[1, 6] = \{1, 2, 2, 1, 5\}$, tiene como máxima frecuencia $2$, y los valores $2$ y $1$ aparecen exactamente $2$ veces. El arreglo $a$ no satisface la condición, pues el valor $1$ aparece $3$ veces, el $2$ aparece $2$ veces, y el $5$ una vez; no hay dos que tengan frecuencia máxima. 
        \end{itemize}
        
        {\itshape Ejemplo 2:}
        \begin{itemize}
            \item El evaluador llama la función 
            
            \begin{center}
                {\itshape Doblemente\_Aficionados(10, \{2, 2, 2, 5, 4, 1, 3, 1, 2, 2\})}
            \end{center}

            \item en este caso, regresar $7$, daría un veredicto de aceptado.
        \end{itemize}
        
        {\itshape Ejemplo 3:}
        \begin{itemize}
            \item El evaluador llama la función 
            
            \begin{center}
                {\itshape Doblemente\_Aficionados(10, \{2, 2, 2, 7, 8, 2, 5, 2, 6, 6\})}
            \end{center}

            \item en este caso, regresar $7$, daría un veredicto de aceptado.
        \end{itemize}

        {\itshape Ejemplo 4:}
        \begin{itemize}
            \item El evaluador llama la función 

            \begin{center}
                {\itshape Doblemente\_Aficionados(10, \{1, 1, 1, 4, 4, 4, 5, 5, 5, 5\})}
            \end{center}

            \item en este caso, regresar $9$, daría un veredicto de aceptado.
        \end{itemize}
        
        {\itshape Ejemplo 5:}
        \begin{itemize}
            \item El evaluador llama la función 

            \begin{center}
                {\itshape Doblemente\_Aficionados(1, \{1\})}
            \end{center}

            \item en este caso, regresar $0$, daría un veredicto de aceptado.
        \end{itemize}
        
    \hh{Consideraciones}
        \begin{itemize}
            \item $1 \le N \le 2 \times 10^5$.
            \item Para todo $0 \le i \le N - 1$, se cumple que $1 \le a[i] \le N$.
            \item Sea $S_N$ la suma de los valores de $N$ sobre todas las llamadas a la función en un caso. Se cumple que $S_N \le 2\times10^5$.
        \end{itemize}
    
    \hh{Subtareas}


    \begin{itemize}
        \item (10 puntos) $N \le 2000$.
        \item (20 puntos) Se cumple que existe un sub arreglo del tamaño de la respuesta, cuyos valores con frecuencia máxima son $1$ y $2$.
        \item (30 puntos) Para todo $0 \le i \le N - 1$, se cumple que $1 \le a[i] \le min(N, 100)$.
        \item (40 puntos) Sin restricciones adicionales.
    \end{itemize}
\end{document}
