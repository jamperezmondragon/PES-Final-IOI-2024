\documentclass[12pt]{scrartcl}
\usepackage{config}
\usepackage{minted}

%\newcommand\mrh{\color{white}\bfseries}
\newcommand\mrc[1]{\begin{tabular}{@{}l@{}} #1 \end{tabular}}
\setlength\arrayrulewidth{0.8pt}

\usemintedstyle{pastie}

\begin{document}
    \hh{Arbol-stían}
    
    
    \vspace{10pt}

    
    \hh{Problema}
    
        Te es dado un entero $N$ y $N - 1$ aristas con pesos. Estas aristas conectan $N$ vértices de tal forma que exista un camino entre cualesquiera dos vértices (es decir, forman un árbol). Para cada camino, definimos su peso como el producto de la cantidad de aristas en él, y el {\bfseries máximo común divisor} de cada uno de los pesos en el camino. Determina el camino simple (que no repite aristas) con peso máximo.
    
    \hh{Detalles de Implementación}

        Debes implementar la función $Arbol\-stian()$. Esta función recibe un entero $N$ y 3 vectores $u, v$ y $w$, cada uno con $N - 1$ elementos. para cada $0 \le i \le N - 2$, $u[i]$ y $v[i]$ son los vértices que se conectan con la arista $i$, y $w[i]$ es su peso. Esta función debe regresar un entero, el peso máximo en camino del árbol.
        La función se vería así:

\begin{minted}{c++}
#include <bits/stdc++.h>
using namespace std;

long long Arbol-stian(int N, vector<int> u, vector<int> v, vector<int> w) {
    // Implementa esta función.
}
    
\end{minted}

    \hh{Límites}
        \begin{itemize}
            \item $1 \le N \le 2\times10^5$.
            \item Los vectores $u, v$ y $w$ tendrán exactamente $N - 1$ elementos.
            \item Para cada $0 \le i \le N - 2$, se cumple que $1 \le u[i] \neq v[i] \le N$. 
            \item Para cada $0 \le i \le N - 2$, se cumple que $0 \le w[i] \le 10^9$.
            \item Se garantiza que el grafo formado por las aristas es un árbol.
        \end{itemize}
    
    \hh{Subtareas}


    \begin{itemize}
        \item (5 puntos) $N \le 2000$.
        \item (15 puntos) Para todo $0 \le i \le N - 2$, se cumple que $w[i] \le 1$.
        \item (20 puntos) Para todo $0 \le i \le N - 2$, se cumple que $u[i] = i + 1, v[i] = i + 2$.
        \item (25 puntos) Para todo $0 \le i \le N - 2$, se cumple que $w[i]$ es una potencia de 2.
        \item (35 puntos) Sin restricciones adicionales.
    \end{itemize}
\end{document}
