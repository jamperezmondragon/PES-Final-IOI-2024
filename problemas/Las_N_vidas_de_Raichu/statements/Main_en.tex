\documentclass[12pt]{scrartcl}
\usepackage{config}
\usepackage{minted}
\usepackage{array}
\usepackage{multirow}
\usepackage{multicol}
\usepackage{amsmath}

%\newcommand\mrh{\color{white}\bfseries}
\newcommand\mrc[1]{\begin{tabular}{@{}l@{}} #1 \end{tabular}}
\setlength\arrayrulewidth{0.8pt}

\usemintedstyle{pastie}

\begin{document}
    \hh{The $N$ Lives of Raichu}
    
    {\itshape This is a communication problem}
    
    \vspace{10pt}

        Raichu is a very friendly and affectionate cat who has lived many lives. But the end has come, and you must help Raichu determine which is the last of his lives.

    \hh{Problem}

        The jury has a binary array $a$ of size $M$, initially all values are $0$. You must implement the function {\itshape Raichu()}. Your program will be executed $N$ times, and each time, the jury will call the function {\itshape Raichu()} with the parameters $N$ and $M$. Your program can call the functions {\itshape modificar()} and {\itshape leer()}. With these functions, you can modify the value of one of the positions of array $a$, or read a position of the array $a$. The function {\itshape Raichu()} returns an integer and must return 1 only on the $N$-th execution of the program, and 0 otherwise. The only information you have is the array $a$. Help Raichu determine which is its $N$-th and last life.

    \hh{Implementation Details}

        You must implement the function {\itshape Raichu()}. This function receives two integers, $N$ and $M$, and returns an integer. Additionally, you can call the function {\itshape leer()}, which receives an integer $0 \le i \le M - 1$ and returns the current value of $a[i]$, and the function {\itshape modificar()} which receives an integer $0 \le i \le M - 1$ and an integer $0 \le x \le 1$, and makes $a[i] = x$.
        To carry out the interaction, you must include the library \textit{``Raichu.h"} with the command \textit{\#include ``Raichu.h"}.

        Here's an example of how the program would look:

\begin{minted}{c++}

#include "Raichu.h"
#include <bits/stdc++.h>
using namespace std;

int Raichu(int N, int M) {
    // Implement this function.
}
    
\end{minted}
    

    \hh{Evaluation Criteria}

        Each test case consists of $N, M$ and ten fixed evaluation parameters $a_1, a_2, \cdots, a_{10}$.

        If your program does not return 1 exactly on the last execution, and 1 on all other executions, you will receive 0 points.

        For each test case, we define $C_1$ as the maximum number of different positions in the binary array that you involve each time your program is executed. 
        
        We say that during an execution you involved position $i$, if you called the function {\itshape modificar()} or {\itshape leer()} with position $i$ (if you call both it still counts as only one position). 
        
        We also define $C_2$, as the maximum number of different positions in the binary array that you involve each time your program is executed, not counting the first execution.
        \newline

        For example, say $N = 2$. If your program called \textit{leer(1)}, \textit{leer(2)} and \textit{modificar(2)} in the first execution, and \textit{leer(1)} in the second, the value of $C_1$ would be 2, and $C_2$ would be 1.
        
        If your program effectively solves the problem, you will receive points based on:
        
        
        \begin{itemize}
            \item In the first test case, you will receive one point for each evaluation parameter $a_i$ that satisfies $C_1 \le a_i$.
            \item In the second test case, you will receive 1 point for each evaluation parameter $a_i$ that satisfies $C_1 \le a_i$, and 1 point for each one that satisfies $C_2 \le a_i$.
            \item In the third test case, you will receive 2 points for each evaluation parameter $a_i$ that satisfies $C_1 \le a_i$, and 1 point for each one that satisfies $C_2 \le a_i$.
            \item In the fourth test case, you will receive 2 points for each evaluation parameter $a_i$ that satisfies $C_1 \le a_i$, and 2 points for each one that satisfies $C_2 \le a_i$.
            
        \end{itemize}


        In the following table, the parameters and specific considerations for each test case are listed:
        
        \null
        
        \begin{tabular}{|c|c|c|c|*{10}{c|}c|}
            \hline
            \textbf{Test Case} & \textbf{Points} & $N$ & $M$ & $a_1$ & $a_2$ & $a_3$ & $a_4$ & $a_5$ & $a_6$ & $a_7$ & $a_8$ & $a_9$ & $a_{10}$ & \textbf{Time} \\
            \hline
            1 & 10 & $2^{10}$ & 10 & 10 & 10 & 10 & 10 & & 10 & 10 & 10 & 10 & 10 & 3s \\
            \cline{1-8} \cline{10-15}
            2 & 20 & $2^{16}$ & \multirow{2}{*}{} & \multirow{2}{*}{} & \multirow{2}{*}{} & \multirow{2}{*}{} & \multirow{2}{*}{} & \multirow{2}{*}{10} & \multirow{2}{*}{} & \multirow{2}{*}{} & \multirow{2}{*}{} & \multirow{2}{*}{} & \multirow{2}{*}{} & 4s \\
            \cline{1-3} \cline{15-15}
            3 & 30 & $2^{20}$ & $10^5$ & 14 & 13 & 12 & 11 & & 9 & 8 & 7 & 6 & 6 & 5s \\
            \cline{1-3} \cline{15-15}
            4 & 40 & $2^{26}$ & \multirow{2}{*}{} & \multirow{2}{*}{} & \multirow{2}{*}{} & \multirow{2}{*}{} & \multirow{2}{*}{} & \multirow{2}{*}{} & \multirow{2}{*}{} & \multirow{2}{*}{} & \multirow{2}{*}{} & \multirow{2}{*}{} & \multirow{2}{*}{} & 20s \\
            \hline
        \end{tabular}

        
\end{document}
