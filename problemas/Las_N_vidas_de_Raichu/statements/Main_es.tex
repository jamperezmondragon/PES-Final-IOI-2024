\documentclass[12pt]{scrartcl}
\usepackage{config}
\usepackage{minted}
\usepackage{array}
\usepackage{multirow}
\usepackage{multicol}
\usepackage{amsmath}

%\newcommand\mrh{\color{white}\bfseries}
\newcommand\mrc[1]{\begin{tabular}{@{}l@{}} #1 \end{tabular}}
\setlength\arrayrulewidth{0.8pt}

\usemintedstyle{pastie}

\begin{document}
    \hh{Las $N$ vidas de Raichu}
    
    {\itshape Este es un problema de Comunicación}
    
    \vspace{10pt}

        Raichu es un gato muy simpático y cariñoso, que ha vivido muchas vidas. Pero ha llegado el final, y debes ayudar a Raichu a determinar cuál es la última de sus vidas. 

    \hh{Problema}

        El jurado tiene un arreglo binario $a$ de tamaño $M$, inicialmente todos los valores son $0$. Debes implementar la función {\itshape Raichu()} . Tu programa será ejecutado $N$  veces, y cada vez, el jurado llamará la función {\itshape Raichu()} con los parámetros $N$ y $M$. Tu programa puede llamar las funciones {\itshape modificar() } y {\itshape leer()}. con dichas funciones, puedes modificar el valor de una de las posiciones de $a$, o leer una posición del arreglo $a$. La función {\itshape Raichu()} regresa un booleano, y debe regresar verdadero únicamente en la $N$-ésima ejecución del programa. La única información que tienes es el arreglo $a$. Ayuda a Raichu a determinar cuál es su $N$-ésima y última vida.


    \hh{Detalles de Implementación}

        Debes implementar la función {\itshape Raichu()}. Esta función recibe dos enteros, $N$ y $M$, y regresa un booleano. Además, puedes llamar la función {\itshape leer()}, que recibe un entero $0 \le i \le M - 1$ y regresa el valor actual de $a[i]$, y la función {\itshape modificar()} que recibe un entero $0 \le i \le M - 1$ y un entero $0 \le x \le 1$, y hace que $a[i] = x$.
        Para llevar a cabo la interacción, debes agregar la librería \textit{``Raichu.h" } con el comando \textit{\#include ``Raichu.h" }. 

        Un ejemplo de cómo se vería el programa es el siguiente:

\begin{minted}{c++}

#include "Raichu.h"
#include <bits/stdc++.h>
using namespace std;

bool Raichu(int N, int M) {
    // Implementa esta función.
}
    
\end{minted}
        

    \hh{Criterios de Evaluación}

        Cada caso de prueba consiste de $N, M$ y diez parámetros de evaluación fijos $a_1, a_2, \cdots, a_{10}$.

        Si tu programa no regresa verdadero exactamente en la última ejecución, y falso en todas las demás, obtendrás 0 puntos.

        Durante un caso de prueba, definimos $C_1$ como la máxima cantidad de posiciones distintas del arreglo binario que involucras en cada una de las veces que se ejecuta tu programa. 
        
        Decimos que durante una ejecución involucraste la posición $i$, si llamaste la función {\itshape modificar()} o {\itshape leer()} con la posición $i$ (si llamas las dos sigue contando como solamente una posición). 
        
        Definamos también, $C_2$, como la máxima cantidad de posiciones distintas del arreglo binario que involucras a partir de la segunda vez que se ejecuta tu programa.
        
        Si tu programa resuelve efectivamente el problema, obtendrás puntos basado en:
        
        
        \begin{itemize}
            \item En el primer caso de prueba, obtendrás un punto por cada parámetro de evaluación $a_i$ que cumpla con $C_1 \le a_i$.
            \item En el segundo caso de prueba, obtendrás 1 punto por cada parámetro de evaluación $a_i$ que cumpla con $C_1 \le a_i$, y 1 punto por cada uno que cumpla $C_2 \le a_i$.
            \item En el tercer caso de prueba, obtendrás 2 puntos por cada parámetro de evaluación $a_i$ que cumpla con $C_1 \le a_i$, y 1 punto por cada uno que cumpla $C_2 \le a_i$.
            \item En el cuarto caso de prueba, obtendrás 2 puntos por cada parámetro de evaluación $a_i$ que cumpla con $C_1 \le a_i$, y 2 puntos por cada uno que cumpla $C_2 \le a_i$.
            
        \end{itemize}


        En la siguiente tabla, están los parámetros y consideraciones específicas para cada caso de prueba:
        
        \null
        
        \begin{tabular}{|c|c|c|c|*{10}{c|}c|}
            \hline
            \textbf{caso} & \textbf{puntos} & $N$ & $M$ & $a_1$ & $a_2$ & $a_3$ & $a_4$ & $a_5$ & $a_6$ & $a_7$ & $a_8$ & $a_9$ & $a_{10}$ & \textbf{tiempo} \\
            \hline
            1 & 10 & $2^{10}$ & 10 & 10 & 10 & 10 & 10 & & 10 & 10 & 10 & 10 & 10 & 3s \\
            \cline{1-8} \cline{10-15}
            2 & 20 & $2^{16}$ & \multirow{2}{*}{} & \multirow{2}{*}{} & \multirow{2}{*}{} & \multirow{2}{*}{} & \multirow{2}{*}{} & \multirow{2}{*}{10} & \multirow{2}{*}{} & \multirow{2}{*}{} & \multirow{2}{*}{} & \multirow{2}{*}{} & \multirow{2}{*}{} & 4s \\
            \cline{1-3} \cline{15-15}
            3 & 30 & $2^{20}$ & $10^5$ & 14 & 13 & 12 & 11 & & 9 & 8 & 7 & 6 & 6 & 5s \\
            \cline{1-3} \cline{15-15}
            4 & 40 & $2^{26}$ & \multirow{2}{*}{} & \multirow{2}{*}{} & \multirow{2}{*}{} & \multirow{2}{*}{} & \multirow{2}{*}{} & \multirow{2}{*}{} & \multirow{2}{*}{} & \multirow{2}{*}{} & \multirow{2}{*}{} & \multirow{2}{*}{} & \multirow{2}{*}{} & 14s \\
            \hline
        \end{tabular}
     

        
\end{document}
