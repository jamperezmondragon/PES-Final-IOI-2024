\documentclass[12pt]{scrartcl}
\usepackage{config}
\usepackage{minted}

%\newcommand\mrh{\color{white}\bfseries}
\newcommand\mrc[1]{\begin{tabular}{@{}l@{}} #1 \end{tabular}}
\setlength\arrayrulewidth{0.8pt}

\usemintedstyle{pastie}

\begin{document}
    \hh{Cynthia-Arbol}
    
    
    \vspace{10pt}

    
    \hh{Problema}
    
        Te es dado un entero $N$ y $N - 1$ aristas. Estas aristas conectan $N$ vértices de tal forma que exista un camino entre cualesquiera dos vértices (es decir, forman un árbol). La distancia entre dos vértices es definida como la mínima cantidad de aristas en un camino entre ellos. Cada vértice tiene asignado un valor de salto. Indiquemos el valor de salto del vértice $i$ con $a[i]$. En un movimiento, Cynthia puede moverse del vértice en el que está, digamos $v$, a cualquier vértice a distancia $a[v]$ de $v$. Determina la cantidad máxima de vértices distintos que puede visitar Cynthia, independientemente de donde comience a moverse. 
        
        
    \hh{Detalles de Implementación}

        Debes implementar la función $Cynthia\_Arbol()$. Esta función recibe un entero $N$ y 3 vectores, $u, v$, con $N - 1$ elementos, y $a$, con $N$ elementos. para cada $0 \le i \le N - 2$, $u[i]$ y $v[i]$ son los vértices que se conectan con la arista $i$. Para cada $0 \le i \le N - 1$, $a[i]$ es el tamaño del salto del vértice con índice $i$. Esta función debe regresar un entero, la cantidad máxima de vértices distintos que es posible visitar.
        La función se vería así:

\begin{minted}{c++}
#include <bits/stdc++.h>
using namespace std;

int Cyntha_Arbol(int N, vector<int> u, vector<int> v, vector<int> a) {
    // Implementa esta función.
}
    
\end{minted}

    \hh{Límites}
        \begin{itemize}
            \item $1 \le N \le 2\times10^5$.
            \item Los vectores $u, v$ y $w$ tendrán exactamente $N - 1$ elementos.
            \item Para cada $0 \le i \le N - 2$, se cumple que $1 \le u[i] \neq v[i] \le N$. 
            \item Para cada $0 \le i \le N - 2$, se cumple que $0 \le a[i] \le 10^9$.
            \item Se garantiza que el grafo formado por las aristas es un árbol.
        \end{itemize}
    
    \hh{Subtareas}


    \begin{itemize}
    
        \item (5 puntos) Para todo $0 \le i \le N - 2$, se cumple que $u[i] = i + 1, v[i] = i + 2$. Además, para todo $0 \le i \le N - 1$, se cumple que $a[i]$ es igual a algún valor $K$. 
        \item (25 puntos) Para todo $0 \le i \le N - 2$, se cumple que $u[i] = i + 1, v[i] = \left\lfloor \frac{i}{2} \right\rfloor$. Además, para todo $0 \le i \le N - 1$, se cumple que $a[i]$ es igual a algún valor $K$. 
        \item (25 puntos) $N \le 2000$.
        \item (45 puntos) Sin restricciones adicionales.
    \end{itemize}
\end{document}
