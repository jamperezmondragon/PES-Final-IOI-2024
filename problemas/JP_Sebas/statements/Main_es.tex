\documentclass[12pt]{scrartcl}
\usepackage{config}
\usepackage{minted}

%\newcommand\mrh{\color{white}\bfseries}
\newcommand\mrc[1]{\begin{tabular}{@{}l@{}} #1 \end{tabular}}
\setlength\arrayrulewidth{0.8pt}

\usemintedstyle{pastie}

\begin{document}
    \hh{JP y sus alumnos}
    
    \hh{Problema}

        A JP le tocó organizar un entrenamiento del PES. En este momento está llevando a cabo una actividad, en la que ordena a los $N$ alumnos en un círculo, $N$ es par. Les da un valor $a_i$ a cada alumno, tal que para cualesquiera dos alumnos adyacentes en el círculo $a_i$ y $a_{i + 1}$ (los índices se toman módulo $N$), se cumpla que $\lvert a_i - a_{i + 1} \lvert \ = 1$. Sebas quiere saber si existen dos alumnos, en posiciones diametralmente opuestas, tal que tengan el mismo valor. Es decir, si existe un entero $i$ tal que $a_{i} = a_{i + \frac{N}{2}}$. Para determinar esto, Sebas puede preguntarle su número a algún alumno. Pero esto es demasiado fácil para él, por lo que te retó a resolver su problema, en a lo más 60 preguntas.
    
    \hh{Detalles de Implementación}

       Debes implementar la función {\itshape Reto\_Sebas()}. Esta función recibe un entero $N$, la cantidad de alumnos. La función debe regresar un valor $i$, que cumpla que $a_i = a_{i + \frac{N}{2}}$, o, si no existe dicho valor, regresa -1. Durante tu programa, podrás llamar la función {\itshape valor()}, esta función recibe un entero $0 \le i \le N - 1$, y regresa el valor de $a_i$.  Para poder llevar a cabo la interacción, debes incluir la librería \textit{``alumnos.h"}, con el comando \textit{\#include ``alumnos.h"}.
       Un ejemplo de cómo se vería el programa, es el siguiente:
    

\begin{minted}{c++}
#include "alumnos.h"
#include <bits/stdc++.h>
using namespace std;

int Reto_Sebas(int N) {
    // Implementa esta función.
}
\end{minted}

    \hh{Ejemplo}
    
        {\itshape Ejemplo 1:}
        \begin{itemize}
            \item El evaluador llama la función 

            \begin{center}
                {\itshape Reto\_Sebas(8)}
            \end{center}

            \item En este caso, el arreglo de alumnos es el siguiente {\itshape \{0, 1, 2, 3, 2, 1, 0, -1\}}.
            \item Un ejemplo de la interacción podría ser el siguiente:
            \begin{center}
                \begin{tabular}{|c|c|}
                    \hline
                     Función llamada & Respuesta \\
                     \hline
                     {\itshape valor(0)} & 0 \\
                     \hline
                     {\itshape valor(1)} & 1 \\
                     \hline
                     {\itshape valor(2)} & 2 \\
                     \hline
                     {\itshape valor(3)} & 3 \\
                     \hline
                     {\itshape valor(4)} & 2 \\
                     \hline
                     {\itshape valor(5)} & 1 \\
                     \hline
                     {\itshape valor(6)} & 0 \\
                     \hline
                     {\itshape valor(7)} & -1 \\
                     \hline
                \end{tabular}
            \end{center}
            \item Y una respuesta que daría un veredicto de aceptado, sería $1$, pues $a_1 = a_5$.
        \end{itemize}

        
        
    \hh{Consideraciones}
        \begin{itemize}
            \item $1 \le N \le 10^5$.
            \item Para todo $0 \le i \le N - 1$, se cumple que $-10^9 \le a[i] \le 10^9$.
            \item Para todo $0 \le i \le N - 1$, se cumple que $\lvert a[i] - a[(i + 1)\%N] \lvert \ = 1$ .
            \item Si llamas la función {\itshape valor()} más de 60 veces, recibirás 0 puntos en ese caso.
        \end{itemize}
    
    \hh{Subtareas}


    \begin{itemize}
        \item (5 puntos) $N \le 60$.
        \item (25 puntos) Recibirás todos los puntos de esta subtarea si regresas -1 cuando no existe la respuesta, y cualquier entero no negativo cuando sí.
        \item (25 puntos) Se garantiza que el arreglo $a$ es creciente y luego decreciente.
        \item (45 puntos) Sin restricciones adicionales.
    \end{itemize}
\end{document}
