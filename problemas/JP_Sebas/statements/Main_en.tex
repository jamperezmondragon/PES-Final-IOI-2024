\documentclass[12pt]{scrartcl}
\usepackage{config}
\usepackage{minted}

%\newcommand\mrh{\color{white}\bfseries}
\newcommand\mrc[1]{\begin{tabular}{@{}l@{}} #1 \end{tabular}}
\setlength\arrayrulewidth{0.8pt}

\usemintedstyle{pastie}

\begin{document}
    \hh{JP and his students}
    
    \hh{Problem}

        JP was tasked with organizing national camp. He is currently conducting an activity where he arranges the $N$ students in a circle, with $N$ even. He gives a value $a_i$ to each student, such that for any two adjacent students in the circle $a_i$ and $a_{i + 1}$ (indices are taken modulo $N$), it holds that $\lvert a_i - a_{i + 1} \lvert \ = 1$. Sebas wants to know if there are two students, in diametrically opposed positions, with the same value. That is, if there exists an integer $i$ such that $a_{i} = a_{i + \frac{N}{2}}$. In order to find out, Sebas may ask any student their number. But this too easy for him, so he challenged you to solve his problem in at most 60 queries.
    
    \hh{Implementation Details}

       You must implement the function {\itshape Reto\_Sebas()}. This function receives an integer $N$, the number of students. The function should return a value $i$ that satisfies $a_i = a_{i + \frac{N}{2}}$, or, if no such value exists, return -1. During your program, you can call the function {\itshape valor()}, which receives an integer $0 \le i \le N - 1$ and returns the value of $a_i$. To carry out the interaction, you must include the library \textit{``alumnos.h"} with the command \textit{\#include ``alumnos.h"}.
       An example of how the program would look is as follows:
    

\begin{minted}{c++}
#include "alumnos.h"
#include <bits/stdc++.h>
using namespace std;

int Reto_Sebas(int N) {
    // Implement this function.
}
\end{minted}

    The grader will call the function \textbf{multiple} times for each case.

    \hh{Example}
    
        {\itshape Example 1:}
        \begin{itemize}
            \item The grader calls the function 

            \begin{center}
                {\itshape Reto\_Sebas(8)}
            \end{center}

            \item In this case, the array of students is the following {\itshape \{0, 1, 2, 3, 2, 1, 0, -1\}}.
            \item An example of the interaction could be as follows:
            \begin{center}
                \begin{tabular}{|c|c|}
                    \hline
                     Function called & Response \\
                     \hline
                     {\itshape valor(0)} & 0 \\
                     \hline
                     {\itshape valor(1)} & 1 \\
                     \hline
                     {\itshape valor(2)} & 2 \\
                     \hline
                     {\itshape valor(3)} & 3 \\
                     \hline
                     {\itshape valor(4)} & 2 \\
                     \hline
                     {\itshape valor(5)} & 1 \\
                     \hline
                     {\itshape valor(6)} & 0 \\
                     \hline
                     {\itshape valor(7)} & -1 \\
                     \hline
                \end{tabular}
            \end{center}
            \item And returning $1$ would give an accepted veredict, since $a_1 = a_5$.
        \end{itemize}

        
        
    \hh{Constraints}
        \begin{itemize}
            \item $1 \le N \le 10^5$.
            \item For all $0 \le i \le N - 1$, it holds that $-10^9 \le a[i] \le 10^9$.
            \item For all $0 \le i \le N - 1$, it holds that $\lvert a[i] - a[(i + 1)\%N] \lvert \ = 1$.
            \item If you call the function {\itshape valor()} more than 60 times during the function {\itshape Reto\_Sebas()}, you will receive 0 points for that case.
            \item Let $S_N$ be the sum of all values of $N$ over each call to the function in a testcase. It is guaranteed that $S_N \le 10^5$.
        \end{itemize}
    
    \hh{Subtasks}


    \begin{itemize}
        \item (5 points) $N \le 60$.
        \item (20 points) It is guaranteed that the array $a$ is increasing and then decreasing.
        \item (35 points) It is guaranteed that there is always an answer.
        \item (40 points) No additional constraints.
    \end{itemize}
\end{document}
