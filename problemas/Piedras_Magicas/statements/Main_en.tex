\documentclass[12pt]{scrartcl}
\usepackage{config}
\usepackage{minted}

%\newcommand\mrh{\color{white}\bfseries}
\newcommand\mrc[1]{\begin{tabular}{@{}l@{}} #1 \end{tabular}}
\setlength\arrayrulewidth{0.8pt}

\usemintedstyle{pastie}

\begin{document}
    \hh{Magic Stones}

    
    \vspace{10pt}

    
    \hh{Problem}

        Given a positive integer $N$, there are $N$ cells in a row, each one unit away from its adjacent cells, numbered from $1$ to $N$. Additionally, there are 2 magic stones, initially located in cells $A$ and $B$. There are $Q$ queries, and after each query, there must be at least one magic stone in a specific position. Moving a magic stone from position $i$ to position $j$ costs $\lvert i - j \lvert$ units. Find the minimum number of cost units needed to satisfy the $Q$ queries, in the given order.
        
    \hh{Implementation Details}

        You must implement the function \textit{Piedras\_Magicas()}. This function receives 4 integers, $N, A, B,$ and $Q$; the number of cells, the initial positions of the magic stones, and the number of queries, respectively. Additionally, it receives a vector $a$, with $Q$ elements. $a[i]$ indicates the position where there must be a magic stone at moment $i$. The function must return a single integer, the minimum number of units needed to satisfy all queries. 
        A sample program would look like this:

\begin{minted}{c++}
#include <bits/stdc++.h>
using namespace std;

long long int Piedras_Magicas(int N, int A, int B, int Q, vector<int> a) {
    // Implement this function.
}
    
\end{minted}

    The grader will call the function \textbf{multiple} times for each test case.
    
    \hh{Examples}
    
        {\itshape Example 1: }
        \begin{itemize}
            \item The grader calls the function 
            \begin{center}
                \textit{Piedras\_Magicas(10, 1, 10, 3, \{3, 6, 1\})}
            \end{center}
            
            \item The function should return $8$. The queries can be processed as follows:
            \begin{itemize}
                \item Move the stone from position 1 to position 3.
                \item Move the stone from position 10 to position 6.
                \item Move the stone from position 3 to position 1.
            \end{itemize}
        \end{itemize}

        {\itshape Example 2:}
        \begin{itemize}
            \item The grader calls the function 
            \begin{center}
                \textit{Piedras\_Magicas(11, 1, 11, 2, \{6, 1\})}
            \end{center}
            
            \item The function should return $5$. The queries can be processed as follows:
            \begin{itemize}
                \item Move the stone from position 11 to position 6.
            \end{itemize}
        \end{itemize}
        
        {\itshape Example 3:}
        \begin{itemize}
            \item The grader calls the function 
            \begin{center}
                \textit{Piedras\_Magicas(11, 1, 11, 2, \{6, 11\})}
            \end{center}
            
            \item The function should return $5$. The queries can be processed as follows:
            \begin{itemize}
                \item Move the stone from position 1 to position 6.
            \end{itemize}
        \end{itemize}
        
        {\itshape Example 4:}
        \begin{itemize}
            \item The grader calls the function 

            \begin{center}
            \textit{Piedras\_Magicas(11, 8, 1, 16, \{1, 1, 5, 1, 11, 4, 5, 2, 5, 3, 3, 3, 5, 5, 6, 7\} )}
            \end{center}
            
            \item The function should return $21$.
        \end{itemize}
        
    \hh{Constraints}
        \begin{itemize}
            \item $1 \le N, Q \le 2\times 10^5$.
            \item $1 \le A, B \le N$.
            \item For all $0 \le i \le Q - 1$, $1 \le a[i] \le N$.
            \item Let $S_N$ be the sum of the values of $N$ over all function calls. It is guaranteed that $S_N \le 2\times10^5$.
            \item Let $S_Q$ be the sum of the values of $Q$ over all function calls. It is guaranteed that $S_Q \le 2\times10^5$.
        \end{itemize}
    
    \hh{Subtasks}

    \begin{itemize}
        \item (10 points) $Q, S_Q \le 20$.
        \item (20 points) $N, S_N, Q, S_Q \le 250$.
        \item (30 points) $N, S_N, Q, S_Q \le 2000$.
        \item (40 points) No additional constraints.
    \end{itemize}
\end{document}
