\documentclass[12pt]{scrartcl}
\usepackage{config}
\usepackage{minted}

%\newcommand\mrh{\color{white}\bfseries}
\newcommand\mrc[1]{\begin{tabular}{@{}l@{}} #1 \end{tabular}}
\setlength\arrayrulewidth{0.8pt}

\usemintedstyle{pastie}

\begin{document}
    \hh{Piedras Mágicas}

    
    \vspace{10pt}

    
    \hh{Problema}

        Dado un entero positivo $N$, hay $N$ casillas en una fila, cada una a distancia 1 de sus adyacentes, numeradas de $1$ a $N$. Además, hay 2 piedras mágicas, que inicialmente están en las casillas $A$ y $B$. Hay $Q$ pedidos, en cada uno, se requiere que haya al menos una piedra mágica en una posición específica. Mover una piedra mágica de la posición $i$ a la posición $j$ cuesta $\lvert i - j \lvert$ unidades. Encuentra la cantidad mínima de unidades necesarias para satisfacer los $Q$ pedidos, en el orden establecido. 
        
    \hh{Detalles de Implementación}

        Debes implementar la función \textit{Piedras\_Magicas()}. Esta función recibe 4 enteros, $N, A, B$ y $Q$; la cantidad de casillas, las posiciones iniciales de las piedras mágicas y la cantidad de pedidos. Además, recibe un vector $a$, con $Q$ elementos. $a[i]$ indica la posición en la que debe haber una piedra mágica en el momento $i$. La función debe regresar un solo entero, la cantidad mínima de unidades necesarias para satisfacer todos los pedidos. 
        Un ejemplo de programa se vería así:

\begin{minted}{c++}
#include <bits/stdc++.h>
using namespace std;

long long int Piedras_Magicas(int N, int A, int B, int Q, vector<int> a) {
    // Implementa esta función.
}
    
\end{minted}

    El evaluador llamará la función \textbf{múltiples} veces por cada caso de prueba.
    
    \hh{Ejemplos}
    
        {\itshape Ejemplo 1: }
        \begin{itemize}
            \item El evaluador llama la función 
            \begin{center}
                \textit{Piedras\_Magicas(10, 1, 10, 3, \{3, 6, 1\})}
            \end{center}
            
            \item La función debe regresar $8$. Los pedidos pueden ser procesados de la siguiente manera:
            \begin{itemize}
                \item Mover la piedra de la posición 1 a la 3.
                \item Mover la piedra de la posición 10 a la 6.
                \item Mover la piedra de la posición 3 a la 1.
            \end{itemize}
        \end{itemize}

        {\itshape Ejemplo 2:}
        \begin{itemize}
            \item El evaluador llama la función 
            \begin{center}
                \textit{Piedras\_Magicas(11, 1, 11, 2, \{6, 1\})}
            \end{center}
            
            \item La función debe regresar $5$. Los pedidos pueden ser procesados de la siguiente manera:
            \begin{itemize}
                \item Mover la piedra de la posición 11 a la posición 6.
            \end{itemize}
        \end{itemize}
        
        {\itshape Ejemplo 3:}
        \begin{itemize}
            \item El evaluador llama la función 
            \begin{center}
                \textit{Piedras\_Magicas(11, 1, 11, 2, \{6, 11\})}
            \end{center}
            
            \item La función debe regresar $5$. Los pedidos pueden ser procesados de la siguiente manera:
            \begin{itemize}
                \item Mover la piedra de la posición 1 a la posición 6.
            \end{itemize}
        \end{itemize}
        
        {\itshape Ejemplo 4:}
        \begin{itemize}
            \item El evaluador llama la función 

            \begin{center}
            \textit{Piedras\_Magicas(11, 8, 1, 16, \{1, 1, 5, 1, 11, 4, 5, 2, 5, 3, 3, 3, 5, 5, 6, 7\} )}
            \end{center}
            
            \item La función debe regresar $21$.
        \end{itemize}
        
    \hh{Consideraciones}
        \begin{itemize}
            \item $1 \le  N, Q \le 2\times 10^5$.
            \item $1 \le A, B \le N$.
            \item Para todo $0 \le i \le Q - 1$, $1 \le a[i] \le N$.
            \item Sea $S_N$ la suma de los valores de $N$ sobre todas las llamadas a la función. Se garantiza que $S_N \le 2\times10^5$.
            \item Sea $S_Q$ la suma de los valores de $Q$ sobre todas las llamadas a la función. Se garantiza que $S_Q \le 2\times10^5$.
        \end{itemize}
    
    \hh{Subtareas}

    \begin{itemize}
        \item (10 puntos) $Q, S_Q \le 20$.
        \item (20 puntos) $N, S_N, Q, S_Q \le 250$.
        \item (30 puntos) $N, S_N, Q, S_Q \le 2000$.
        \item (40 puntos) Sin restricciones adicionales.
    \end{itemize}
\end{document}
