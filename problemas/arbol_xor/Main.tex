\documentclass[12pt]{scrartcl}
\usepackage{config}
\usepackage{minted}

%\newcommand\mrh{\color{white}\bfseries}
\newcommand\mrc[1]{\begin{tabular}{@{}l@{}} #1 \end{tabular}}
\setlength\arrayrulewidth{0.8pt}

\usemintedstyle{pastie}

\begin{document}
    \hh{Árbol Xor}
    
    
    \vspace{10pt}

    
    \hh{Problema}
    
        Te es dado un entero $N$ y $N - 1$ aristas con pesos. Estas aristas conectan $N$ vértices de tal forma que exista un camino entre cualesquiera dos vértices (es decir, forman un árbol). Para cada camino, definimos su peso como el {\bfseries xor } de cada uno de los pesos en el camino. Determina la suma de los pesos de todos los caminos simples (no repiten aristas) del árbol.
    
    \hh{Detalles de Implementación}

        Debes implementar la función $Encuentra\_xor()$. Esta función recibe un entero $N$, 3 vectores $u, v$ y $w$, cada uno con $N - 1$ elementos. para cada $0 \le i \le N - 2$, $u[i]$ y $v[i]$ son los vértices que se conectan con la arista $i$, y $w[i]$ es su peso. Esta función debe regresar un entero, la suma de los pesos de todos los caminos.
        La función se vería así:

\begin{minted}{c++}
#include <bits/stdc++.h>
using namespace std;

long long Encuentra_xor(int N, vector<int> u, vector<int> v, vector<int> w) {
    // Implementa esta función.
}
    
\end{minted}

    \hh{Límites}
        \begin{itemize}
            \item $1 \le N \le 2\times10^5$.
            \item Los vectores $u, v$ y $w$ tendrán exactamente $N - 1$ elementos.
            \item Para cada $0 \le i \le N - 2$, se cumple que $1 \le u[i] \neq v[i] \le N$. 
            \item Para cada $0 \le i \le N - 2$, se cumple que $0 \le w[i] \le 10^9$.
            \item Se garantiza que el grafo formado por las aristas es un árbol.
        \end{itemize}
    
    \hh{Subtareas}


    \begin{itemize}
        \item (10 puntos) $N \le 2000$.
        \item (20 puntos) Para todo $0 \le i \le N - 2$, se cumple que $w[i] \le 1$.
        \item (25 puntos) Para todo $0 \le i \le N - 2$, se cumple que $u[i] = i + 1, v[i] = i + 2$.
        \item (45 puntos) Sin restricciones adicionales.
    \end{itemize}
\end{document}
