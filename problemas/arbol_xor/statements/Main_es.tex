\documentclass[12pt]{scrartcl}
\usepackage{config}
\usepackage{minted}

%\newcommand\mrh{\color{white}\bfseries}
\newcommand\mrc[1]{\begin{tabular}{@{}l@{}} #1 \end{tabular}}
\setlength\arrayrulewidth{0.8pt}

\usemintedstyle{pastie}

\begin{document}
    \hh{Árbol Xor}
    
    
    \vspace{10pt}

    
    \hh{Problema}
    
        Te es dado un entero $N$ y $N - 1$ aristas con pesos. Estas aristas conectan $N$ vértices de tal forma que exista un camino entre cualesquiera dos vértices (es decir, forman un árbol).
        Un camino simple en un grafo es definido como una secuencia de $k$ vértices $\{v_1, v_2, \cdots , v_k\}$, tal que para todo $1 \le i \le k - 1$, la arista $\{v_i, v_{i + 1}\}$ existe en el grafo. 
        
        Para cada camino, definimos su peso como el {\bfseries xor} (or exclusivo, aquí consideramos la operación bit por bit) de cada uno de los pesos de las aristas que componen el camino. Determina la suma de los pesos de todos los caminos simples (no repiten aristas) del árbol (el camino $\{a, b\}$ se considera el mismo que $\{b, a\}$).
        
    \hh{Detalles de Implementación}

        Debes implementar la función $Encuentra\_xor()$. Esta función recibe un entero $N$, 3 vectores $u, v$ y $w$, cada uno con $N - 1$ elementos. para cada $0 \le i \le N - 2$, $u[i]$ y $v[i]$ son los vértices que se conectan con la arista $i$, y $w[i]$ es su peso. Esta función debe regresar un entero, la suma de los pesos de todos los caminos.
        La función se vería así:

\begin{minted}{c++}
#include <bits/stdc++.h>
using namespace std;

long long Encuentra_xor(int N, vector<int> u, vector<int> v, vector<int> w) {
    // Implementa esta función.
}
    
\end{minted}

    El evaluador correrá una vez tu programa por cada caso.

    \hh{Ejemplos}

               
        {\itshape Ejemplo 1:}
        
        \begin{itemize}
            \item  El evaluador llama la función 
            $$Encuentra\_xor(5, \{0, 1, 0, 4\}, \{1, 2, 3, 1\}, \{2, 3, 4, 0\})$$ 
            el árbol en este caso se ilustra en la siguiente imagen:
            
            \begin{center}
                \includegraphics[scale=0.3]{arbol_xor/ej1.png}
            \end{center}
            
            \item los xors de los caminos son:

                \begin{center}
                    
                \begin{tabular}{|c||c|c|c|c|c|}
                     \hline
                      $\oplus$ & 0 & 1 & 2 & 3 & 4  \\
                     \hline
                     \hline 
                     0 & 0 & 2 & 1 & 4 & 2 \\
                     \hline 
                     1 & 2 & 0 & 3 & 6 & 0 \\ 
                     \hline
                     2 & 1 & 3 & 0 & 5 & 3 \\
                     \hline
                     3 & 4 & 6 & 5 & 0 & 6 \\
                     \hline
                     4 & 2 & 0 & 3 & 6 & 0 \\
                     \hline
                \end{tabular}
                
                \end{center}
            \item  La función debe regresar 32, la suma del xor de todos los caminos (el camino $\{a, b\}$ se considera el mismo que $\{b, a\}$).
                
        \end{itemize}


        {\itshape Ejemplo 2:}
        \begin{itemize}
            \item El evaluador llama la función $$Encuentra\_xor(9, \{0, 1, 0, 1, 0, 2, 3, 3\}, \{1, 2, 3, 4, 5, 6, 7, 8\}, \{2, 3, 4, 5, 1, 0, 7, 2\})$$ el arbol en este caso es el siguiente:

            \begin{center}
                \includegraphics[scale=0.5]{arbol_xor/ej2.png}
            \end{center}
            \item La función debe regresar 132.
            
        \end{itemize}
               
        


    \eject
    
    \hh{Límites}
        \begin{itemize}
            \item $1 \le N \le 2\times10^5$.
            \item Los vectores $u, v$ y $w$ tendrán exactamente $N - 1$ elementos.
            \item Para cada $0 \le i \le N - 2$, se cumple que $0 \le u[i] \neq v[i] < N$. 
            \item Para cada $0 \le i \le N - 2$, se cumple que $0 \le w[i] \le 10^9$.
            \item Se garantiza que el grafo formado por las aristas es un árbol.
        \end{itemize}
    
    \hh{Subtareas}


    \begin{itemize}
        \item (10 puntos) $N \le 2000$.
        \item (20 puntos) Para todo $0 \le i \le N - 2$, se cumple que $w[i] \le 1$.
        \item (25 puntos) Para todo $0 \le i \le N - 2$, se cumple que $u[i] = i, v[i] = i + 1$.
        \item (45 puntos) Sin restricciones adicionales.
    \end{itemize}
\end{document}
